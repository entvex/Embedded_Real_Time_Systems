\section{Exercise 3.3 - Consumer and producer with TCP}

This exercise models a producer and consumer thread by two independent modules. They exchange messages (in this case TCP packets) via a SystemC \textit{sc\_fifo} queue, with the producer transmitting a new packet every 2-10$MS$. The consumer receives the packet on its queue and prints out the sequence number and time stamp. The definition of the \textit{TCPHeader} is found in the exercise. We extended the model to include multiple consumers receiving packets on different ports.

\noindent Listing \ref{lst:tcpheadermain} shows the application file for our producer/consumer setup. We create one producer, two consumers and then two \textit{sc\_fifo} channels for each consumer. The outbound channel of the producer is set to both channels.

\begin{lstlisting}[style=customc++, caption=Application file for producer/consumer.,
label={lst:tcpheadermain}]
int sc_main(int argc, char* argv[]) {

	producer my_producer("producer");
	consumer my_consumer1("consumer1");
	consumer my_consumer2("consumer2");

	sc_fifo<TCPHeader*> C1("C1");
	sc_fifo<TCPHeader*> C2("C2");

	my_producer.out(C1);
	my_producer.out(C2);
	my_consumer1.in(C1);
	my_consumer2.in(C2);

	sc_start(2000,SC_MS);
	return 0;
}
\end{lstlisting}
\newpage
\noindent The producer internally uses a \textit{sc\_port} class to bind the two channel interfaces. The zero parameter allows binding a unlimited number of interfaces. We can access each binded channel using the index operator as shown in the \textit{producer\_thread} and write to it. The producer waits a random time interval, creates a new TCP packet, set a sequence number and writes that to each binded channel. See listing \ref{lst:tcpheaderproducer}.

\begin{lstlisting}[style=customc++, caption=Implementation of TCP producer.,
label={lst:tcpheaderproducer}]
SC_MODULE(producer){
	sc_port<sc_fifo_out_if<TCPHeader*>,0> out;
	int min = 2;
	int max = 10;
	int counter = 0;

	void producer_thread(void) {
		while(true) {
			int random = rand() % min+max;
			wait(random,SC_MS);
			TCPHeader* newpacket = new TCPHeader();
			counter++;
			newpacket->SequenceNumber = counter;

			for(int i = 0; i < out.size(); i++) {
				out[i]->write(newpacket);
			}
		}
	}
SC_CTOR (producer) {
	SC_THREAD(producer_thread)
}
};
\end{lstlisting}

\noindent The consumer receives the TCP packets in a thread and prints out the time stamp and the sequence number. Note the \textit{object} variable is a pointer to a \textit{TCPHeader} type, hence the arrow notation to dereference it. See listing \ref{lst:tcpheaderconsumer}.

\begin{lstlisting}[style=customc++, caption=Implementation of TCP consumer.,
label={lst:tcpheaderconsumer}]
SC_MODULE(consumer){
	sc_fifo_in<TCPHeader*> in;

	void consumer_thread(void) {
	while(true) {
		TCPHeader* object = in->read();
		std::cout << "Timestamp: " << sc_time_stamp() <<
		" - Seq: " << object->SequenceNumber <<
		" " << name() << std::endl;
		}
	}
SC_CTOR (consumer) {
	SC_THREAD(consumer_thread)
}
};
\end{lstlisting}
