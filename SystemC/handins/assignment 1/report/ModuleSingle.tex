\section{Exercise 3.1 - ModuleSingle}

We implemented the application with a main.cpp and a ModuleSingle.h file, containing the headers and the implementation code. The main.cpp starts the simulation by instantiating the Single class and calling \textit{sc\_start()} running for 200$MS$. This is shown in listing \ref{lst:modulesinglemain}

\begin{lstlisting}[style=customc++, caption=Application file for ModuleSingle,
label={lst:modulesinglemain}]
#include <iostream>
using namespace std;
#include ""
#include "ModuleSingle.h"
int sc_main(int argc, char* argv[]) {
	Single ModuleSingle("my_instance 1");
	sc_start(200,SC_MS);
	return 0;
}
\end{lstlisting}

\noindent The ModuleSingle.h defines the Single module functionality that performs the exercise task. It contains a single thread \textit{my\_thread\_process} and a single method \textit{my\_method}. The method is static sensitive to the event \textit{my\_event}. Inside the method is simply a counter that increments a \textit{sc\_uint<4>} variable when invoked and outputs the value using cout. The thread executes in a while-loop that notifies the method every 2$MS$ by calling the event \textit{my\_event}. The counter variable is just 4 bits, so it resets when it wraps around (16 or higher). The code for ModuleSingle.h is shown in listing \ref{lst:modulesingleheader}.

\begin{lstlisting}[style=customc++, caption=Implementation of ModuleSingle,
label={lst:modulesingleheader}]
#include <systemc.h>
SC_MODULE(Single) {
	sc_event my_event;
	sc_uint<4> counter;
	void my_thread_process(void) {
		while(true) {
			my_event.notify();
			wait(2,SC_MS);
		}
	}
	void my_method(void) {
		counter++;
		std::cout << "counter: " << counter << "
		- Timestamp: " << sc_time_stamp() << std::endl;
	}
	
	SC_CTOR (Single) {
		counter = 0;
		SC_THREAD(my_thread_process)
			sensitive << my_event;
		SC_METHOD(my_method)
			sensitive << my_event;
	} };
\end{lstlisting}