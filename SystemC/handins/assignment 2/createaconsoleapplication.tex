\section{Exercise 3 Create a Console Application}
In this exercise we use the hardware and software platform that we made in exercise 2. In this lab we write a C-application that implements a command language interpreter, controlled via the USB-UART interface.

With the following commands

\begin{itemize}
	\item Sets the binary value from 0-15 on the red led’s by reading switch input (SW0-SW3) 
	\item Counts binary the red led’s using a timer of 1 sec. 
\end{itemize}

Here we read the values from the switches and it out on the leds.

\begin{lstlisting}[style=customc++, caption= read switchs and set value on leds,
label={lst:masterslaveadaptersource}]
	   switch (value) {
case 49: // ASCII 1
xil_printf("case 1 \r\n");
psb_check = XGpio_DiscreteRead(&push, 1);
xil_printf("Push Buttons Status %x\r\n", psb_check);
dip_check = XGpio_DiscreteRead(&dip, 1);
xil_printf("DIP Switch Status %x\r\n", dip_check);
LED_IP_mWriteReg(XPAR_LED_IP_S_AXI_BASEADDR,0, dip_check);
break;
\end{lstlisting}

To get the counter running we need to make some setup code first.

\begin{lstlisting}[style=customc++, caption= setup timer,
label={lst:masterslaveadaptersource}]
   //TimerSetup
XScuTimer Timer;
#define ONE_SECOND 325000000
XScuTimer_Config *ConfigPtr;
XScuTimer *TimerInstancePtr = &Timer;

ConfigPtr = XScuTimer_LookupConfig(XPAR_PS7_SCUTIMER_0_DEVICE_ID);
Status    = XScuTimer_CfgInitialize(TimerInstancePtr, ConfigPtr, ConfigPtr->BaseAddr);

if(Status != XST_SUCCESS) {
xil_printf("Timer init() failed\r\n");
return XST_FAILURE;
}

XScuTimer_LoadTimer(TimerInstancePtr,ONE_SECOND);
XScuTimer_EnableAutoReload(TimerInstancePtr);
\end{lstlisting}

\pagebreak
after that setup it's time to handle the extra case. When we fall into this case we start the timer. So that it will be running all the time.

\begin{lstlisting}[style=customc++, caption= case 2: Counts binary the red led’s using a timer of 1 sec ,
label={lst:masterslaveadaptersource}]
		case 50: // ASCII 2
xil_printf("case 2 \r\n");
XScuTimer_Start(TimerInstancePtr);
break;
\end{lstlisting}

Every time the timer elapses it runs the following code that makes the led go from 0 to 15.
\begin{lstlisting}[style=customc++, caption= case 2: Counts binary the red led’s using a timer of 1 sec ,
label={lst:masterslaveadaptersource}]
	   if(XScuTimer_IsExpired(TimerInstancePtr)) {
XScuTimer_ClearInterruptStatus(TimerInstancePtr);
red_led_counter++;
xil_printf("Current count: %d\r\n", red_led_counter);
LED_IP_mWriteReg(XPAR_LED_IP_S_AXI_BASEADDR,0, red_led_counter);
}

for (i=0; i<9999999; i++);
\end{lstlisting}

We can conclude that it works. The leds counts as it should and we learned that into from the serial is ASCII, but that was a easy fix.









